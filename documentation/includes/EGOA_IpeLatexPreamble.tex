%
% EGOA_IpeLatexPreamble.tex
%
%  Created on: Nov 11, 2017
%      Author: Franziska Wegner
%
% Usepackages and includes that are used for mainly all ipe presentations.
%

%%%%%%%%%%%%%%%%%%%%%%%%%%%%%%%%%%%%%%%%%%%%%%%%%%%%%%%%%%%%%%%%%%%%%%%%%%%%%
% Usepackages
%%%%%%%%%%%%%%%%%%%%%%%%%%%%%%%%%%%%%%%%%%%%%%%%%%%%%%%%%%%%%%%%%%%%%%%%%%%%%
%
\usepackage[T1]{fontenc}
\usepackage{lmodern}

\usepackage{tikz,pgfplots}
\pgfplotsset{compat=newest}
\usepgfplotslibrary{polar}

\usepackage{pifont} % for \ding{55} -> cross
\usepackage{amsmath,amssymb,amsfonts}
\usepackage{mathtools}
\usepackage{xspace}
\usepackage{accents}
\usepackage{paralist}

\DeclareMathOperator{\lef}{\ell}
\DeclareMathOperator{\rig}{r}

\usepackage{amsopn}
\usepackage{xargs}       % Use more than one optional parameter in a new commands

\usepackage{scrhack}

\usepackage[ngerman,main=american]{babel}

\usepackage{calc}

\usepackage{printlen}

\usepackage{nicefrac}
\usepackage{xfrac}
\usepackage[super]{nth} %\nth{1}, \nth{2}, \nth{3}, \nth{4}

% SI units
\usepackage[binary-units=true]{siunitx}
\sisetup{
    % quotient-mode = fraction,
    per = fraction,
    % fraction-function=\sfrac,
    % product-units = single,
    % output-product={\,},
    unit-color = UNITCOLOR,
    % round-mode=places,
    % round-precision=2,
    % exponent-product = \cdot,
    % output-decimal-marker = {.},
    fraction=sfrac
}

\usepackage{complexity}

% Table
\usepackage{tabularx}
\usepackage{multirow}
\usepackage{booktabs}
\def\sharedaffiliation{%
  \end{tabular}
  \begin{tabular}{c}
}
\usepackage{colortbl}
\usepackage{pdflscape}
\usepackage{floatrow}
\floatsetup[table]{capposition=top}

% Quotes
\usepackage{csquotes}
\usepackage{currvita}


\makeatletter%https://tex.stackexchange.com/questions/265841/small-overleftrightarrow
    \newcommand{\overleftrightsmallarrow}{\mathpalette{\overarrowsmall@\leftrightarrowfill@}}
    \newcommand{\overrightsmallarrow}{\mathpalette{\overarrowsmall@\rightarrowfill@}}
    \newcommand{\overleftsmallarrow}{\mathpalette{\overarrowsmall@\leftarrowfill@}}
%
    \newcommand{\overarrowsmall@}[3]{%
    \vbox{%
      \ialign{%
        ##\crcr
        #1{\smaller@style{#2}}\crcr
        \noalign{\nointerlineskip}%
        $\m@th\hfil#2#3\hfil$\crcr
      }%
    }%
    }
    \def\smaller@style#1{%
    \ifx#1\displaystyle\scriptstyle\else
      \ifx#1\textstyle\scriptstyle\else
        \scriptscriptstyle
      \fi
    \fi
    }
\makeatother
%
%%%%%%%%%%%%%%%%%%%%%%%%%%%%%%%%%%%%%%%%%%%%%%%%%%%%%%%%%%%%%%%%%%%%%%%%%%%%%
% Includes
%%%%%%%%%%%%%%%%%%%%%%%%%%%%%%%%%%%%%%%%%%%%%%%%%%%%%%%%%%%%%%%%%%%%%%%%%%%%%
\input{/Users/fwegner/Documents/work/dissertation/03-Document/EGOA_Macros.tex}

\usepackage[most]{tcolorbox}
\usepackage{transparent}
% 
%%%%%%%%%%%%%%%%%%%%%%%%%%%%%%%%%%%%%%%%%%%%%%%%%%%%%%%%%%%%%%%%%%%%%%%%%%%%%
% KIT defined colors
%%%%%%%%%%%%%%%%%%%%%%%%%%%%%%%%%%%%%%%%%%%%%%%%%%%%%%%%%%%%%%%%%%%%%%%%%%%%%
% 
% KIT green
\definecolor{KITgreen}{rgb}{0 0.588 0.509}
\definecolor{KITgreen70}{rgb}{0.3 0.711 0.656}
\definecolor{KITgreen50}{rgb}{0.5 0.794 0.754}
\definecolor{KITgreen30}{rgb}{0.7 0.876 0.852}
\definecolor{KITgreen15}{rgb}{0.85 0.938 0.926}
% KIT palegreen
\definecolor{KITpalegreen}{rgb}{0.509,0.745,0.235}
\definecolor{KITpalegreen70}{rgb}{0.656,0.821,0.464}
\definecolor{KITpalegreen50}{rgb}{0.754,0.872,0.617}
\definecolor{KITpalegreen30}{rgb}{0.852,0.923,0.77}
\definecolor{KITpalegreen15}{rgb}{0.926,0.961,0.885}
% KIT yellow
\definecolor{KITyellow}{rgb}{0.98,0.901,0.078}
\definecolor{KITyellow70}{rgb}{0.986,0.931,0.354}
\definecolor{KITyellow50}{rgb}{0.99,0.95,0.539}
\definecolor{KITyellow30}{rgb}{0.994,0.97,0.723}
\definecolor{KITyellow15}{rgb}{0.997,0.985,0.861}
% KIT orange
\definecolor{KITorange}{rgb}{0.862,0.627,0.117}
\definecolor{KITorange70}{rgb}{0.903,0.739,0.382}
\definecolor{KITorange50}{rgb}{0.931,0.813,0.558}
\definecolor{KITorange30}{rgb}{0.958,0.888,0.735}
\definecolor{KITorange15}{rgb}{0.979,0.944,0.867}
% KIT brown
\definecolor{KITbrown}{rgb}{0.627,0.509,0.196}
\definecolor{KITbrown70}{rgb}{0.739,0.656,0.437}
\definecolor{KITbrown50}{rgb}{0.813,0.754,0.598}
\definecolor{KITbrown30}{rgb}{0.888,0.852,0.758}
\definecolor{KITbrown15}{rgb}{0.944,0.926,0.879}
% KIT red
\definecolor{KITred}{rgb}{0.627,0.117,0.156}
\definecolor{KITred70}{rgb}{0.739,0.382,0.409}
\definecolor{KITred50}{rgb}{0.813,0.558,0.578}
\definecolor{KITred30}{rgb}{0.888,0.735,0.747}
\definecolor{KITred15}{rgb}{0.944,0.867,0.873}
% KIT lilac
\definecolor{KITlilac}{rgb}{0.627,0,0.47}
\definecolor{KITlilac70}{rgb}{0.739,0.3,0.629}
\definecolor{KITlilac50}{rgb}{0.813,0.5,0.735}
\definecolor{KITlilac30}{rgb}{0.888,0.7,0.841}
\definecolor{KITlilac15}{rgb}{0.944,0.85,0.92}
% KIT blue
\definecolor{KITblue}{rgb}{0.274,0.392,0.666}
\definecolor{KITblue70}{rgb}{0.492,0.574,0.766}
\definecolor{KITblue50}{rgb}{0.637,0.696,0.833}
\definecolor{KITblue30}{rgb}{0.782,0.817,0.9}
\definecolor{KITblue15}{rgb}{0.891,0.908,0.95}
% KIT seablue
\definecolor{KITseablue}{rgb}{0.196,0.313,0.549}
\definecolor{KITseablue70}{rgb}{0.437,0.519,0.684}
\definecolor{KITseablue50}{rgb}{0.598,0.656,0.774}
\definecolor{KITseablue30}{rgb}{0.758,0.794,0.864}
\definecolor{KITseablue15}{rgb}{0.879,0.897,0.932}
% KIT cyanblue
\definecolor{KITcyanblue}{rgb}{0.313,0.666,0.901}
\definecolor{KITcyanblue70}{rgb}{0.519,0.766,0.931}
\definecolor{KITcyanblue50}{rgb}{0.656,0.833,0.95}
\definecolor{KITcyanblue30}{rgb}{0.794,0.9,0.97}
\definecolor{KITcyanblue15}{rgb}{0.897,0.95,0.985}
% KIT black
\definecolor{KITblack}{rgb}{0,0,0}
\definecolor{KITblack70}{rgb}{0.3,0.3,0.3}
\definecolor{KITblack50}{rgb}{0.5,0.5,0.5}
\definecolor{KITblack30}{rgb}{0.7,0.7,0.7}
\definecolor{KITblack15}{rgb}{0.85,0.85,0.85}
% 
%%%%%%%%%%%%%%%%%%%%%%%%%%%%%%%%%%%%%%%%%%%%%%%%%%%%%%%%%%%%%%%%%%%%%%%%%%%%%
% Power Grid Related Colors
%%%%%%%%%%%%%%%%%%%%%%%%%%%%%%%%%%%%%%%%%%%%%%%%%%%%%%%%%%%%%%%%%%%%%%%%%%%%%
%
% Power Grid specific colors
\definecolor{EGOA_Feasible}{rgb}{0.739,0.3,0.629}
\definecolor{EGOA_Physical}{rgb}{0.437,0.519,0.684}
\definecolor{EGOA_Kcl}{rgb}{0.739,0.3,0.629}
\definecolor{EGOA_Kvl}{rgb}{0.437,0.519,0.684}
% Problems
\definecolor{EGOA_Mtsf}{rgb}{0.509,0.745,0.235}
\definecolor{EGOA_Mpf}{rgb}{0.437,0.519,0.684}
\definecolor{EGOA_Mf}{rgb}{0.739,0.3,0.629}
\definecolor{EGOA_Ots}{rgb}{0,0,0}
\definecolor{EGOA_Rop}{rgb}{0,0,0}
\definecolor{EGOA_Mff}{rgb}{0,0,0}
\definecolor{EGOA_Opf}{rgb}{0,0,0}
\definecolor{EGOA_Changes}{rgb}{0.997,0.985,0.861}
% Elements
\definecolor{EGOA_Generator}{rgb}{0.437,0.519,0.684}
\definecolor{EGOA_Consumer}{rgb}{0.739,0.3,0.629}
\definecolor{EGOA_Susceptance}{rgb}{0.862,0.627,0.117}
\definecolor{EGOA_Switches}{rgb}{0,0.588,0.509}
\definecolor{EGOA_VoltageAngle}{rgb}{0.313,0.666,0.901}
\definecolor{EGOA_Capacity}{rgb}{0.5,0.5,0.5}
\definecolor{EGOA_Dtp}{rgb}{0.627,0.117,0.156}

\colorlet{Table-Line-Marker}    {KITyellow15}
\colorlet{EGOA_DualGraph}       {EGOA_VoltageAngle}
\colorlet{EGOA_PrimalGraph}     {KITblack50}
\colorlet{EGOA_KvlConflict}     {KITred70}
\colorlet{EGOA_Substation}      {KITbrown70}
\colorlet{EGOA_Voltage}         {KITblue70}
\colorlet{EGOA_Current}         {KITred70}
\colorlet{EGOA_RealPower}       {KITcyanblue70}
\colorlet{EGOA_ReactivePower}   {KITcyanblue30}
\colorlet{EGOA_ComplexPower}    {KITgreen70}
\colorlet{EGOA_PowerAngle}      {KITpalegreen70}
\colorlet{EGOA_Timestamp}       {KITblack70}
\colorlet{EGOA_UnitColor}       {KITblack70}
\colorlet{EGOA_ColorTableRule}  {KITblack70}
\colorlet{EGOA_ColorKclConflictMarker}       {KITred70}
\colorlet{EGOA_ColorSusceptanceScalingMarker}{EGOA_Susceptance}
\colorlet{EGOA_ColorFactsEdge}  {KITred70}

% Wind farm cabling
\colorlet{ColorTransmissionCable}       {KITseablue}
\colorlet{ColorSubstation}              {KITbrown}
\colorlet{ColorCollectionPoint}         {ColorSubstation}
\colorlet{ColorTransportCableSmall}     {KITblack50}
\colorlet{ColorTransportCableMedium}    {KITblack70}
\colorlet{ColorTransportCableLarge}     {KITblack}
\colorlet{ColorCircuitProblem}          {KITgreen}
\colorlet{ColorSubstationProblem}       {KITorange}
\colorlet{ColorFullWindFarm}            {KITred}
% 
%%%%%%%%%%%%%%%%%%%%%%%%%%%%%%%%%%%%%%%%%%%%%%%%%%%%%%%%%%%%%%%%%%%%%%%%%%%%%
% Other 
%%%%%%%%%%%%%%%%%%%%%%%%%%%%%%%%%%%%%%%%%%%%%%%%%%%%%%%%%%%%%%%%%%%%%%%%%%%%%
% ACM already defined colors
\definecolor[named]{ACMBlue}{cmyk}{1,0.1,0,0.1}
\definecolor[named]{ACMYellow}{cmyk}{0,0.16,1,0}
\definecolor[named]{ACMOrange}{cmyk}{0,0.42,1,0.01}
\definecolor[named]{ACMRed}{cmyk}{0,0.90,0.86,0}
\definecolor[named]{ACMLightBlue}{cmyk}{0.49,0.01,0,0}
\definecolor[named]{ACMGreen}{cmyk}{0.20,0,1,0.19}
\definecolor[named]{ACMPurple}{cmyk}{0.55,1,0,0.15}
\definecolor[named]{ACMDarkBlue}{cmyk}{1,0.58,0,0.21}
% Helmholtz
\definecolor{HELMHOLTZblue}{rgb}{0.043,0.362,0.618}
\definecolor{HELMHOLTZgreen}{rgb}{0.549,0.7,0.19}
% Others
\definecolor{orange}{HTML}{FF7F00}
\definecolor{orange}{rgb}{1,0.5,0}
\definecolor{orange}{RGB}{255,127,0}
\definecolor{figChar}{cmyk}{0,0.90,0.86,0}
% 
\input{/Users/fwegner/Documents/work/dissertation/03-Document/EGOA_TikzCommands.tex}

%%%%%%%%%%%%%%%%%%%%%%%%%%%%%%%%%%%%%%%%%%%%%%%%%%%%%%%%%%%%%%%%%%%%%%%%%%%%%
% Problem Environment
%%%%%%%%%%%%%%%%%%%%%%%%%%%%%%%%%%%%%%%%%%%%%%%%%%%%%%%%%%%%%%%%%%%%%%%%%%%%%
%
\makeatletter
% \renewcommand\arraystretch{1.5}
\newenvironment{problem}[2][]{%
  \def\problem@arg{#1}%
  \def\problem@framed{framed}%
  \def\problem@lined{lined}%
  \def\problem@doublelined{doublelined}%
  \ifx\problem@arg\@empty%
    \def\problem@hline{}%
  \else%
    \ifx\problem@arg\problem@doublelined%
      \def\problem@hline{\hline\hline}%
    \else%
      \def\problem@hline{\hline}%
      \def\problem@hrule{\hrule}%
    \fi%
  \fi%
  \ifx\problem@arg\problem@framed%
    \def\problem@table{\centering\tabularx{1\columnwidth}{|>
{\itshape}rX|c}}%
    \def\problem@title{\multicolumn{2}{|l|}{%
        \raisebox{-\fboxsep}{\textsc{#2}}%
      }}%
  \else
    \def\problem@table{\tabularx{1\columnwidth}{>{\bfseries}rXc}}%
    \def\problem@title{\multicolumn{2}{l}{%
        \raisebox{-\fboxsep}{\textsc{\large #2}}%
      }}%
  \fi%
  \bigskip\par\noindent%
  \renewcommand{\arraystretch}{1.2}%
    \problem@table%
      \problem@hline%
      \problem@title\\[2\fboxsep]\hspace*{0.5cm}%%
}{%
      % \\\problem@hline%
    \endtabularx%
    \problem@hrule
  \medskip\par%
}
\makeatother

%%%%%%%%%%%%%%%%%%%%%%%%%%%%%%%%%%%%%%%%%%%%%%%%%%%%%%%%%%%%%%%%%%%%%%%%%%%%%
% Algorithm Environment
%%%%%%%%%%%%%%%%%%%%%%%%%%%%%%%%%%%%%%%%%%%%%%%%%%%%%%%%%%%%%%%%%%%%%%%%%%%%%
%
\usepackage[titlenumbered, ruled, vlined, linesnumbered,resetcount]
{algorithm2e}%, dotocloa
\SetKwComment{Comment}{\color{KITblack30}$\triangleright$\ }{}
\SetCommentSty{tiny}
\SetKw{kwnot}{not}
\SetKw{kwand}{and}
\SetKw{kwdominates}{dominates}
\SetKwFunction{deleteMin}{delMin}
\SetKwFunction{update}{update}
\SetKwFunction{myinsert}{insert}
\SetKwFunction{getPaths}{getPaths}
\SetKwFunction{deleteDominatedLabels}{deleteDominatedLabels}
\SetKwFunction{isReachable}{isReachable}
\SetKwFunction{algodfs}{dfs}
\SetKwFunction{algoplanarembedding}{planarEmbeddingOf}
\SetKwFunction{algoconstructDualGraphOf}{constructDualGraphOf}
\SetKwFunction{algoBipolarSubgraphOf}{bipolarSubgraphOf}
\SetKwFunction{algoDualGraphOf}{dualGraphOf}
\SetKwFunction{algoMinCostFlow}{minCostFlow}
% \DontPrintSemicolon
\SetKwFunction{topologicSort}{topologicSort}
\SetKwFunction{computeColors}{computeColors}
\SetKw{kwfrom}{from}
\SetKw{kwto}{to}
\SetKw{kwcontinue}{continue}

\newcommand{\assign}{\longleftarrow}

\SetAlFnt{\tiny}
\SetAlTitleFnt{\tiny}
\SetAlCapFnt{\tiny}
\SetAlCapNameFnt{\tiny}
\newcommand{\commentfont}[1]{{\color{black50}#1}}
\newcommand{\kwfont}[1]{\texttt{\color{stroke1}#1}}
\newcommand{\funcfont}[1]{\texttt{#1}}
\newcommand{\datafont}[1]{\texttt{#1}}

\SetCommentSty{commentfont}
\SetKwSty{kwfont}
\SetFuncSty{funcfont}
\SetDataSty{datafont}

\SetKwFunction{queueIsEmpty}{isEmpty}%
\SetKwFunction{queueIsNotEmpty}{isNotEmpty}%
\SetKwFunction{queueMinElement}{minElement}%
\SetKwFunction{queueDeleteMin}{deleteMin}%
\SetKwFunction{queueDeleteMax}{deleteMax}%
\SetKwFunction{queueInsert}{insert}%
\SetKwFunction{queueUpdate}{update}%
\SetKwFunction{setInsert}{insert}%
\SetKwFunction{setDeleteDominatedLabels}{deleteLabelsDominatedBy}%
\SetKw{KwNotDominates}{does not dominate}%

\makeatletter
\newcommand{\setalgotoprulecolor}[1]{\colorlet{toprulecolor}{#1}}
\let\old@algocf@pre@ruled\@algocf@pre@ruled
\renewcommand{\@algocf@pre@ruled}{\textcolor{toprulecolor}{\old@algocf@pre@ruled}}

\newcommand{\setalgobotrulecolor}[1]{\colorlet{bottomrulecolor}{#1}}
\let\old@algocf@post@ruled\@algocf@post@ruled
\renewcommand{\@algocf@post@ruled}{\textcolor{bottomrulecolor}{\old@algocf@post@ruled}}

\renewcommand{\algocf@caption@ruled}{\box\algocf@capbox\vspace{-1mm}
\textcolor{bottomrulecolor}{\old@algocf@post@ruled}}%
\makeatother

\setalgotoprulecolor{stroke1}
\setalgobotrulecolor{stroke1}
